\documentclass{article}
\usepackage{amsmath,theorem,listings}
\usepackage[usenames]{xcolor}
\usepackage[margin=1in,letterpaper]{geometry}
\usepackage[noend]{algpseudocode}

\usepackage[pdftex,
colorlinks=true,
plainpages=false, % only if colorlinks=true
linkcolor=blue,   % only if colorlinks=true
citecolor=blue,   % only if colorlinks=true
urlcolor=blue     % only if colorlinks=true
]{hyperref}

\newcommand{\tmop}[1]{\ensuremath{\operatorname{#1}}}
{ \theorembodyfont{\upshape} \newtheorem{note}{Note} }
{ \theorembodyfont{\upshape} \newtheorem{remark}{Remark} }

\lstset{
language=Python,
basicstyle=\footnotesize\ttfamily,
keywordstyle=\color{blue},
numbers=left,
numbersep=5pt,
numberstyle=\tiny\color{gray},
commentstyle=\color{gray},
}

\begin{document}

\title{Solving the Poisson equation in 3D using $Q_{1}$ finite elements}
\author{C.Khroulev}
\maketitle

{\tableofcontents}

\section{Model problem}

I will focus on implementing a $Q_1$ FEM solver for the 3D Poisson equation on
a ``cube'', with Dirichlet boundary conditions on 5 sides and Neumann on the
remaining one.

\subsection{Domain}

\begin{eqnarray*}
  \Omega & = & (- 1, 1) \times (- 1, 1) \times (- 1, 1),\\
  \partial \Omega & = & \{ (x, y, z) | x = \pm 1 \vee y = \pm 1 \vee z = \pm
  1 \},\\
  \partial \Omega_N & = & \partial \Omega \setminus \{ (x, y, z) |z = 1 \},\\
  \partial \Omega_D & = & \partial \Omega \setminus \partial \Omega_N .
\end{eqnarray*}

\subsection{Boundary value problem}

Find a function $u$ satisfying the following: $u \in C^2 (\Omega)$, $u \in
C^0 (\bar{\Omega})$, and
\begin{eqnarray}
  - \nabla^2 u & = & f \text{ in } \Omega,  \label{eq:poisson}\\
  u & = & g_D  \text{ on } \partial \Omega_D, \nonumber\\
  \frac{\partial u}{\partial \vec{n}} & = & g_N  \text{ on } \partial \Omega_N
  . \nonumber
\end{eqnarray}
I will consider the particular case $g_N \equiv 0$, because I don't know how
to approximate boundary integrals otherwise...

\subsection{Weak formulation}

Define solution and test spaces
\begin{eqnarray*}
  \mathcal{H}_E^1 (\Omega) & = & \{ u \in \mathcal{H}^1 (\Omega) | u = g_D
  \text{ on } \partial \Omega_D \},\\
  \mathcal{H}^1_{E_0} (\Omega) & = & \{ u \in \mathcal{H}^1 (\Omega) | u = 0
  \text{ on } \partial \Omega_D \} .
\end{eqnarray*}
Note that by the ``derivative of the product'' rule
\[ \nabla \cdot (v \nabla u) = \nabla v \cdot \nabla u + v \nabla^2 u, \]
or
\begin{equation}
  - v \nabla^2 u = \nabla v \cdot \nabla u - \nabla \cdot (v \nabla u) .
  \label{eq:productrule}
\end{equation}
In addition to this, the divergence theorem applied to the last term in
(\ref{eq:productrule}) implies
\[ \int_{\Omega} \nabla \cdot (v \nabla u) = \int_{\partial \Omega} v \nabla
   u \cdot \vec{n} = \int_{\partial \Omega} v g_N . \]
Using this, I multiply (\ref{eq:poisson}) by a test function $\varphi \in
\mathcal{H}^1_{E_0} (\Omega)$ and integrate by parts to obtain the weak form
\begin{equation}
  \int_{\Omega} \nabla u \cdot \nabla \varphi = \int_{\Omega} f \varphi +
  \int_{\partial \Omega_N} \varphi g_N  \text{ for all } \varphi \in
  \mathcal{H}^1_{E_0} (\Omega) \label{eq:weakpoisson}
\end{equation}

\section{Galerkin approximation}

Define a logically rectangular grid on $\Omega$:
\begin{equation*}
\begin{array}{lllllllllll}
  - 1 & = & X_{0, j, k} & < & X_{i, j, k} & < & X_{i + 1, j, k} & < & X_{I, j, k} & = & 1,\\
  &  &  &  & X_{i + 1, j, k} & - & X_{i, j, k} & = & \Delta x, &  & \\
  - 1 & = & Y_{i, 0, k} & < & Y_{i, j, k} & < & Y_{i, j + 1, k} & < & Y_{i, J, k} & = & 1,\\
  &  &  &  & Y_{i, j + 1, k} & - & Y_{i, j, k} & = & \Delta y, &  & \\
  - 1 & = & Z_{i, j, 0} & < & Z_{i, j, k} & < & Z_{i, j, k + 1} & < & Z_{i, j, K} & = & 1.\\
\end{array}
\end{equation*}
There are no assumptions about the $Z$ spacing at any $i$ and $j$.

This gives $N_P = (I + 1) (J + 1) (K + 1)$ grid points and splits $\Omega$
into $N_E = I \times J \times K$ hexahedral elements.

See \cite[section 1.3]{Elmanetal2005} for the construction of approximating
spaces $S_0^h$ for $\mathcal{H}^1_{E_0}$ and $S_E^h$ for $\mathcal{H}_E^1$
(this procedure is the same for all Galerkin methods).

Given this approximation, (\ref{eq:weakpoisson}) becomes
\begin{equation}
  \begin{array}{c}
    \text{Find } u_h \in S_E^h  \text{ so that}\\
    \displaystyle
    \int_{\Omega} \nabla u_h \cdot \nabla v_h = \int_{\Omega} f v_h +
    \int_{\partial \Omega_N} v_h g_N  \text{ for all } v_h \in S_0^h (\Omega)
    . \label{eq:galerkinpoisson}
  \end{array}
\end{equation}
Neumann boundary conditions appear in (\ref{eq:galerkinpoisson}) explicitly,
while Dirichlet boundary conditions are built into the solution space
$\mathcal{H}_E^1 (\Omega)$.

\subsection{Building the system of equations}

For computation it is convenient to enforce (\ref{eq:galerkinpoisson}) for
each basis function $\varphi_j$. Then (\ref{eq:galerkinpoisson}) is equivalent
to finding $u_i$ so that
\begin{eqnarray*}
  \int_{\Omega} \sum_{i = 1}^N u_i \nabla \varphi_i \cdot \nabla \varphi_j & =
  & \int_{\Omega} f \varphi_j + \int_{\partial \Omega_N} \varphi_j g_N,\\
  \sum_{i = 1}^N u_i \int_{\Omega} \nabla \varphi_i \cdot \nabla \varphi_j & =
  & \int_{\Omega} f \varphi_j + \int_{\partial \Omega_N} \varphi_j g_N .
\end{eqnarray*}
holds for all $j = 1, \ldots, N$.

This can be written as $A u = b$, where $A$ is an $N_P \times N_P$ matrix and
$b$ is a column vector with $N_P$ elements, and
\begin{eqnarray*}
  a_{i, j} & = & \int_{\Omega} \nabla \varphi_i \cdot \nabla \varphi_j,\\
  b_j & = & \int_{\Omega} f \varphi_j + \int_{\partial \Omega_N} \varphi_j g_N
  .
\end{eqnarray*}
Now, the partition of $\Omega$ into elements $E_k$ (and the fact that
$\tmop{supp} (\varphi_i) \cap \tmop{supp} (\varphi_j)$ is either empty or
equal to an element $E_k$ for some $k$) allows re-writing $a_{i, j}$ and $b_j$
in terms of integrals over individual elements:
\begin{eqnarray}
  a_{i, j} & = & \sum_{k = 1}^{N_P} \int_{E_k} \nabla \varphi_i \cdot \nabla
  \varphi_j,  \label{eq:matrix}\\
  b_j & = & \sum_{k = 1}^{N_P} \int_{E_k} f \varphi_j + \int_{\partial
  \Omega_N \cap E_k} g_N \varphi_j .  \label{eq:rhs}
\end{eqnarray}
\begin{note}
  Everything above applies in the case of \emph{any} Galerkin approximation
  of the boundary value problem for the Poisson equation on $\Omega$.
\end{note}

\begin{remark}
  I find definitions of ``global'' basis functions $\varphi_j$ distracting.
  All I really need to know are preimages (with respect to $M$
  (\ref{eq:map})) of restrictions of $\varphi_j$ to individual elements \
  \emph{and} that $\varphi_j$ resulting from out choices of $M$ and $\chi_j$
  have desired smoothness properties.
\end{remark}

\section{$Q_1$ elements}

I define element basis functions by considering the reference element and using
a linear isoparametric map to physical elements. This ensures that
$\varphi_j$ are continuous across element boundaries and produces a conforming
approximation \cite[section 1.3.4]{Elmanetal2005}.

\subsection{Element basis}\label{sec:elementbasis}

Consider the reference element \cite[page 2.126]{Kardestuncer1987}:
\begin{equation}
  E_{\ast} = [ - 1, 1] \times [ - 1, 1] \times [ - 1, 1] \label{eq:reference}
\end{equation}
in the 3D Cartesian right-handed coordinate system with coordinates $\xi, \eta, \zeta$. I list the corners of this hexahedron in the
counter-clockwise order, first for $\zeta = - 1$, then for $\zeta = 1$:
\begin{eqnarray}
  \xi & = & (- 1, 1, 1, - 1, - 1, 1, 1, - 1), \nonumber\\
  \eta & = & (- 1, - 1, 1, 1, - 1, - 1, 1, 1),  \label{eq:order}\\
  \zeta & = & (- 1, - 1, - 1, - 1, 1, 1, 1, 1) . \nonumber
\end{eqnarray}
Now the coordinates of the $j$-th node are $(\xi_j, \eta_j, \zeta_j)$ and all
$8$ shape functions can be defined by
\begin{equation}
  \chi_j (\xi, \eta, \zeta) = \frac{1}{8} (1 + \xi \xi_j) (1 + \eta \eta_j)
  (1 + \zeta \zeta_j) . \label{eq:elementbasis}
\end{equation}

\subsection{Map from the reference to a physical element}

I number the nodes of a physical element $E_k$ in the order defined by
(\ref{eq:order}) and use the \emph{isoparametric} map $M : (\xi, \eta,
\zeta) \mapsto (x, y, z)$:
\begin{eqnarray}
  x (\xi, \eta, \zeta) & = & \sum_{j = 1}^8 x_j \cdot \chi_j (\xi, \eta,
  \zeta), \nonumber\\
  y (\xi, \eta, \zeta) & = & \sum_{j = 1}^8 y_j \cdot \chi_j (\xi, \eta,
  \zeta),  \label{eq:map}\\
  z (\xi, \eta, \zeta) & = & \sum_{j = 1}^8 z_j \cdot \chi_j (\xi, \eta,
  \zeta) . \nonumber
\end{eqnarray}
Let $J_k$ be the Jacobian of $M$,
\begin{equation}
  J_k = \left(\begin{array}{ccc}
    \frac{\partial x}{\partial \xi} & \frac{\partial y}{\partial \xi} &
    \frac{\partial z}{\partial \xi}\\
    \frac{\partial x}{\partial \eta} & \frac{\partial y}{\partial \eta} &
    \frac{\partial z}{\partial \eta}\\
    \frac{\partial x}{\partial \zeta} & \frac{\partial y}{\partial \zeta} &
    \frac{\partial z}{\partial \zeta}
  \end{array}\right) . \label{eq:jacobian}
\end{equation}
\begin{note}
  $J_k$ is a function of coordinates of nodes of the corresponding physical
  element $k$.
\end{note}

\begin{note}
  An important requirement is $| J_k | \neq 0$, which ensures that $M$ is
  invertible.
\end{note}

\begin{remark}
  This is \emph{two} maps, in fact: one mapping the reference element to a
  physical element and the second one mapping element basis functions $\chi_j$
  defined on the reference element to restrictions of ``global'' basis
  functions $\varphi_j$ on physical elements. The term ``isoparametric'' means
  that these two maps have the same number of parameters.
\end{remark}

\section{Assembling the linear system}

My goal is to assemble the linear system defined by (\ref{eq:matrix}) and
(\ref{eq:rhs}). Now is the time to re-write these very general expressions in
Cartesian coordinates and think about approximating integrals that appear in
definitions of $a_{i, j}$ and $b_j$.

\subsection{Assembling the matrix}

In Cartesian coordinates (\ref{eq:matrix}) becomes
\begin{eqnarray}
  a_{i, j} & = & \sum_{k = 1}^{N_P} \int_{E_k} \nabla \varphi_i \cdot \nabla
  \varphi_j \nonumber\\
  & = & \sum_{k = 1}^{N_P} \int_{E_k} \left(\frac{\partial
  \varphi_i}{\partial x}, \frac{\partial \varphi_i}{\partial y},
  \frac{\partial \varphi_i}{\partial z} \right) \cdot \left(\frac{\partial
  \varphi_j}{\partial x}, \frac{\partial \varphi_j}{\partial y},
  \frac{\partial \varphi_j}{\partial z} \right) \tmop{dx} \tmop{dy} \tmop{dz}
  \nonumber\\
  & = & \sum_{k = 1}^{N_P} I_k, \text{ where} \nonumber\\
  I_k & = & \int_{E_k} \left(\frac{\partial \varphi_i}{\partial x},
  \frac{\partial \varphi_i}{\partial y}, \frac{\partial \varphi_i}{\partial z}
  \right) \cdot \left(\frac{\partial \varphi_j}{\partial x}, \frac{\partial
  \varphi_j}{\partial y}, \frac{\partial \varphi_j}{\partial z} \right)
  \tmop{dx} \tmop{dy} \tmop{dz} . \label{eq:ik}
\end{eqnarray}
I would like to approximate $I_k$ using a quadrature, but defining quadrature
points and weights on an arbitrary physical element is unpractical. Instead,
I change variables to express $I_k$ as an integral over $E_{\ast}$ (see
(\ref{eq:reference})) and with respect to $d \xi, d \eta, d \zeta$. Then I
can use a standard quadrature on $E_{\ast}$, usually obtained using a tensor
product of points and weights corresponding to a one-dimensional quadrature
scheme.

\subsubsection{Change of variables}

The chain rule implies
\begin{equation}
\frac{\partial \varphi_j}{\partial x} = \frac{\partial \varphi_j}{\partial
   \xi} \cdot \frac{\partial \xi}{\partial x} + \frac{\partial
   \varphi_j}{\partial \eta} \cdot \frac{\partial \eta}{\partial x} +
   \frac{\partial \varphi_j}{\partial \zeta} \cdot \frac{\partial
   \zeta}{\partial x} \label{eq:chainrule}
 \end{equation}
and similarly for $y$ and $z$. So, on a physical element $E_k$ I have
\begin{equation}
  \left(
    \begin{array}{c}
      \frac{\partial \varphi_j}{\partial x}\\
      \frac{\partial \varphi_j}{\partial y}\\
      \frac{\partial \varphi_j}{\partial z}
   \end{array}
 \right) = J_k^{- 1} \left(
   \begin{array}{c}
     \frac{\partial \varphi_j}{\partial \xi}\\
     \frac{\partial \varphi_j}{\partial \eta}\\
     \frac{\partial \varphi_j}{\partial \zeta}
   \end{array}
 \right),\label{eq:2}
\end{equation}
where $J_k^{- 1}$ is the inverse of the Jacobian (\ref{eq:jacobian}). On the
reference element $E_{\ast}$, though, I have $\varphi_i = \chi_i$, so I get
\begin{equation}
  I_k = \int_{E_{\ast}} J_k^{- 1} \left( \begin{array}{c}
    \frac{\partial \chi_i}{\partial \xi}\\
    \frac{\partial \chi_i}{\partial \eta}\\
    \frac{\partial \chi_i}{\partial \zeta}
  \end{array} \right) \cdot J_k^{- 1} \left( \begin{array}{c}
    \frac{\partial \chi_j}{\partial \xi}\\
    \frac{\partial \chi_j}{\partial \eta}\\
    \frac{\partial \chi_j}{\partial \zeta}
  \end{array} \right) | J_k | d \xi d \eta d \zeta . \label{eq:ikfinal}
\end{equation}


\subsection{Assembling the right hand side vector $b$}

Recall that the right hand side of our system is
\begin{eqnarray}
  b_j & = & \sum_{k = 1}^{N_P} \left(\int_{E_k} f \varphi_j + \int_{\partial
  \Omega_N \cap E_k} g_N \varphi_j \right), \nonumber\\
  & = & \sum_{k = 1}^{N_P} (F_j + N_j), \text{ where} \nonumber\\
  F_j & = & \int_{E_k} f \varphi_j,  \label{eq:force}\\
  N_j & = & \int_{\partial \Omega_N \cap E_k} g_N \varphi_j .
  \label{eq:neumann}
\end{eqnarray}
I will approximate $F_k$ and $N_k$ using quadratures. In all practical cases
analytical expressions for $f$ and $g_{N}$ are no available but their values at locations of grid
(or mesh) nodes are, so I have to approximate integrands first. The easiest and
most natural way to approximate $f$ and $g_N$ is
\begin{eqnarray*}
  f^h & = & \sum_{j = 1}^{N_P} f (p_j) \varphi_j,\\
  g^h_N & = & \sum_{p_j \in \partial \Omega_N} g_N (p_j) \varphi_j .
\end{eqnarray*}
Then
\begin{eqnarray}
  F_j & \approx & \int_{E_k} \varphi_j \sum_{i = 1}^{N_P} f (p_i) \varphi_i
  \nonumber\\
  & = & \sum_{i = 1}^{N_P} \int_{E_k} f (p_i) \varphi_i \varphi_j,
  \label{eq:forcefinal}\\
  & = & \sum_{i = 1}^{N_P} \int_{E_{\ast}} f (p_i) \chi_i \chi_j  | J_k |
  \nonumber\\
  N_j & \approx & \int_{\partial \Omega_N \cap E_k} \varphi_j \sum_{p_j \in
  \partial \Omega_N} g_N (p_j) \varphi_j \nonumber\\
  & = & \sum_{p_j \in \partial \Omega_{N \cap E_k}} \int_{\partial \Omega_N
  \cap E_k} g_N (p_j) \varphi_i \varphi_j .  \label{eq:neumannfinal}
\end{eqnarray}
\begin{note}
  It is just as easy to handle variable-coefficient equations, e.g. $- \nabla
  (k (x, y, z) \nabla u) = f$.
\end{note}

\begin{remark}
  It is worth pointing out that evaluating $f^h$ at a quadrature point $p$ is
  equivalent to using trilinear interpolation and nodal values of $f$ to
  approximate $f (p)$.
\end{remark}

\subsection{Gaussian quadrature on the reference element $E_{\ast}$}

As pointed out earlier one of the advantages of defining shape functions using
a reference element and a linear map (\ref{eq:map}) to a physical element is
that it allows one to use quadratures obtained as a tensor product of 1D
quadratures on the standard interval $[ - 1, 1]$.

To approximate $I_k$ (see (\ref{eq:ikfinal})) and $F_j$ (see
(\ref{eq:forcefinal})) I use the $8$-point quadrature
\begin{eqnarray*}
  q_i = (\xi^{\ast}_i, \eta^{\ast}_i, \zeta^{\ast}_i) & = & \left(
  \frac{\xi_i}{\sqrt{3}}, \frac{\eta_i}{\sqrt{3}}, \frac{\zeta_i}{\sqrt{3}}
  \right),\\
  w_i & = & 1,
\end{eqnarray*}
where $i = 1, \ldots, 8$ and $\xi_i$, $\eta_i$, and $\zeta_i$ are defined in
(\ref{eq:order}).

To approximate $N_j$ I need a 2D quadrature on the \emph{face}
approximating $\partial \Omega_N \cap E_k$. I use the $2 \times 2$ Gaussian
quadrature
\begin{eqnarray*}
  \xi & = & \left(\frac{- 1}{\sqrt{3}}, \frac{1}{\sqrt{3}},
  \frac{1}{\sqrt{3}}, \frac{- 1}{\sqrt{3}} \right),\\
  \eta & = & \left(\frac{- 1}{\sqrt{3}}, \frac{- 1}{\sqrt{3}},
  \frac{1}{\sqrt{3}}, \frac{1}{\sqrt{3}} \right),\\
  \zeta_i & = & 1,\\
  w_i & = & 1.
\end{eqnarray*}
(This corresponds to the top face of an element. I don't know where to go from
here, though.)

\section{Implementation: an outline}

Information above is sufficient to implement a Poisson solver... but a very
inefficient one.

This section describes some implementation techniques that make assembling
large systems feasible.

\subsection{Precompute values of $\chi_i$ and their partial derivatives}

The first thing to notice is that all integrals I need to approximate ($I_k$
--- see (\ref{eq:ikfinal}) and $F_j$ --- see (\ref{eq:forcefinal})) allow
precomputing values of element basis functions and their partial derivatives.
(These integrals involve derivatives with respect to $\xi$, $\eta$, and
$\zeta$, not $x$, $y$, and $z$.)

\subsection{Combine $| J_k |$ with quadrature weights}

The determinant of the Jacobian $J_k$ at a quadrature point always appears
multiplied by the corresponding quadrature weight. (Note also that in the
quadrature I use all weights are equal to $1$.)

\subsection{Loop through the grid (or mesh) element by element}

When assembling the matrix, go element by element and add element
contributions to corresponding matrix elements. This allows one to easily
handle irregular meshes, plus this way each $I_k$ and $F_J$ is computed only
once. See \cite[section 1.4.3]{Elmanetal2005} for details.

A naive version of the assembly code for $A$ and $b$ is in Figure \ref{fig:pythonnaive}.
\begin{figure}
  \centering
  \begin{lstlisting}
for k in xrange(n_elements):
  E.reset(pts[elements[k]])

  for i in xrange(E.n_chi()):
    row = elements[k, i]

    for j in xrange(E.n_chi()):
      col = elements[k, j]

      for q in xrange(n_quadrature_points):
        dchi_i = E.J_inverse(q_pts[q]) * E.dchi(i, q_pts[q])
        dchi_j = E.J_inverse(q_pts[q]) * E.dchi(j, q_pts[q])

        A[row, col] += q_w[q] * E.det_J(q_pts[q]) * (dchi_i.T * dchi_j)

        b[row] += (q_w[q] * E.det_J(q_pts[q]) * E.chi(i, q_pts[q]) *
                   (f[col] * E.chi(j, q_pts[q])))

for k in xrange(n_nodes):
  if bc_nodes[k]:
    A[k,:] = 0.0
    A[k,k] = 1.0
    b[k] = u_bc[k]
  \end{lstlisting}
  \caption{Naive assembly in Python. Compare to \eqref{eq:ikfinal} and \eqref{eq:forcefinal}.}
\label{fig:pythonnaive}
\end{figure}

This code is easy to understand, it is generic (works with various element and quadrature types), but it is very slow.
One can do a lot better by moving some computations out of inner loops. The improved assembly algorithm follows.

\begin{figure}
  \centering
  \begin{lstlisting}
# precompute values of shape functions at all quadrature points:
chi = np.zeros((n_quadrature_points, E.n_chi()))
for j in xrange(E.n_chi()):
    for q in xrange(n_quadrature_points):
        chi[j, q] = E.chi(j, q_pts[q])

# precompute values of derivatives of shape functions at all quadrature points:
dchi = np.zeros((n_quadrature_points, E.n_chi(), 2))
for q in xrange(n_quadrature_points):
    for j in xrange(E.n_chi()):
        dchi[q,j,:] = E.dchi(j, q_pts[q]).T

# for each element...
for k in xrange(n_elements):
    # initialize the current element
    E.reset(pts[elements[k]])

    # compute dchi with respect to x,y (using chain rule)
    # and det(J)*w at all quadrature points:
    dchi_xy = np.zeros((E.n_chi(), n_quadrature_points,  2))
    det_JxW = np.zeros(n_quadrature_points)
    for q in xrange(n_quadrature_points):
        J_inv = E.J_inverse(q_pts[q])
        for j in xrange(E.n_chi()):
            dchi_xy[j,q,:] = (J_inv * np.matrix(dchi[q,j,:]).T).T

        det_JxW[q] = E.det_J(q_pts[q]) * q_w[q]

    # for each shape function $\phi_i$...
    for i in xrange(E.n_chi()):
        row = elements[k, i]

        # for each shape function $\phi_j$...
        for j in xrange(E.n_chi()):
            col = elements[k, j]

            for q in xrange(n_quadrature_points):
                # stiffness matrix:
                A[row, col] += det_JxW[q] * np.dot(dchi_xy[i,q], dchi_xy[j,q])

                # right hand side:
                b[row] += det_JxW[q] * chi[i,q] * (f[col]  * chi[q,j])

# enforce Dirichlet boundary conditions:
for k in xrange(n_nodes):
    if bc_nodes[k]:
        A[k,:] = 0.0
        A[k,k] = 1.0 * scaling
        b[k] = u_bc[k] * scaling
  \end{lstlisting}
  \caption{Precomputing linear system assembly. This code is for the 2D Poisson case;
    only two or three lines need to change to switch to 3D, though.}
  \label{fig:pythonbetter}
\end{figure}
\section{Precomputing shape functions at quadrature points}

For every shape function $\chi_i$ and every quadrature point $q_j$ I compute
$Q_{i, j} = \chi_i (q_j)$. This can be done by hand --- or using a short
Maxima script. With present choices of shape functions and a quadrature I get
\begin{equation}
  Q = \left(\begin{array}{llllllll}
    H^3 & H^2 L & HL^2 & H^2 L & H^2 L & HL^2 & L^3 & HL^2\\
    H^2 L & H^3 & H^2 L & HL^2 & HL^2 & H^2 L & HL^2 & L^3\\
    HL^2 & H^2 L & H^3 & H^2 L & L^3 & HL^2 & H^2 L & HL^2\\
    H^2 L & HL^2 & H^2 L & H^3 & HL^2 & L^3 & HL^2 & H^2 L\\
    H^2 L & HL^2 & L^3 & HL^2 & H^3 & H^2 L & HL^2 & H^2 L\\
    HL^2 & H^2 L & HL^2 & L^3 & H^2 L & H^3 & H^2 L & HL^2\\
    L^3 & HL^2 & H^2 L & HL^2 & HL^2 & H^2 L & H^3 & H^2 L\\
    HL^2 & L^3 & HL^2 & H^2 L & H^2 L & HL^2 & H^2 L & H^3
  \end{array} \right), \label{eq:shapefunctions}
\end{equation}
where
\begin{eqnarray*}
  L & = & \frac{1}{2} \left(1 - \frac{1}{\sqrt{3}} \right),\\
  H & = & \frac{1}{2} \left(1 + \frac{1}{\sqrt{3}} \right) .
\end{eqnarray*}
Similarly, one can use a 3D array to store pre-computed partial derivatives of
shape functions.

Here are derivatives of shape functions with respect to $\xi$, $\eta$, and
$\zeta$ (first column --- first shape function, second column --- second shape
function...):
\begin{equation}
  Q_{\xi} = \left(\begin{array}{llllllll}
    H^2 M & H^2 P & HLP & HLM & HLM & HLP & L^2 P & L^2 M\\
    H^2 M & H^2 P & HLP & HLM & HLM & HLP & L^2 P & L^2 M\\
    HLM & HLP & H^2 P & H^2 M & L^2 M & L^2 P & HLP & HLM\\
    HLM & HLP & H^2 P & H^2 M & L^2 M & L^2 P & HLP & HLM\\
    HLM & HLP & L^2 P & L^2 M & H^2 M & H^2 P & HLP & HLM\\
    HLM & HLP & L^2 P & L^2 M & H^2 M & H^2 P & HLP & HLM\\
    L^2 M & L^2 P & HLP & HLM & HLM & HLP & H^2 P & H^2 M\\
    L^2 M & L^2 P & HLP & HLM & HLM & HLP & H^2 P & H^2 M
  \end{array} \right), \label{eq:diffxi}
\end{equation}
\begin{equation}
  Q_{\eta} = \left(\begin{array}{llllllll}
    H^2 M & HLM & HLP & H^2 P & HLM & L^2 M & L^2 P & HLP\\
    HLM & H^2 M & H^2 P & HLP & L^2 M & HLM & HLP & L^2 P\\
    HLM & H^2 M & H^2 P & HLP & L^2 M & HLM & HLP & L^2 P\\
    H^2 M & HLM & HLP & H^2 P & HLM & L^2 M & L^2 P & HLP\\
    HLM & L^2 M & L^2 P & HLP & H^2 M & HLM & HLP & H^2 P\\
    L^2 M & HLM & HLP & L^2 P & HLM & H^2 M & H^2 P & HLP\\
    L^2 M & HLM & HLP & L^2 P & HLM & H^2 M & H^2 P & HLP\\
    HLM & L^2 M & L^2 P & HLP & H^2 M & HLM & HLP & H^2 P
  \end{array} \right), \label{eq:diffeta}
\end{equation}
\begin{equation}
  Q_{\zeta} = \left(\begin{array}{llllllll}
    H^2 M & HLM & L^2 M & HLM & H^2 P & HLP & L^2 P & HLP\\
    HLM & H^2 M & HLM & L^2 M & HLP & H^2 P & HLP & L^2 P\\
    L^2 M & HLM & H^2 M & HLM & L^2 P & HLP & H^2 P & HLP\\
    HLM & L^2 M & HLM & H^2 M & HLP & L^2 P & HLP & H^2 P\\
    H^2 M & HLM & L^2 M & HLM & H^2 P & HLP & L^2 P & HLP\\
    HLM & H^2 M & HLM & L^2 M & HLP & H^2 P & HLP & L^2 P\\
    L^2 M & HLM & H^2 M & HLM & L^2 P & HLP & H^2 P & HLP\\
    HLM & L^2 M & HLM & H^2 M & HLP & L^2 P & HLP & H^2 P
  \end{array} \right) . \label{eq:diffzeta}
\end{equation}
Here $P = \frac{1}{2}$ and $M = - \frac{1}{2}$. For example,
\[ Q_{\xi, 1, 3} = \frac{\partial \chi_3}{\partial \xi} (\xi_1, \eta_1,
   \zeta_1) = \frac{1}{2}  \frac{\left(1 + \frac{1}{\sqrt{3}} \right)}{2}
   \frac{\left(1 - \frac{1}{\sqrt{3}} \right)}{2} = HLP. \]
\begin{remark}
  The only reason to list these here is to better understand (and document)
  Jed's FEM code.
\end{remark}

\section{Pre-computing the Jacobian at quadrature points}

Although it is possible to compute the Jacobian $J_k$, its determinant, and
its inverse using (\ref{eq:map}) alone, I want to take advantage of any
special cases to avoid unnecessary computations.

\subsection{Special case: equal spacing in horizontal directions}

In our application I assume that the spacing in the $x$ and $y$ directions is
fixed and equal to $\Delta x$ and $\Delta y$ respectively. Then
\[ J_k = \left(\begin{array}{ccc}
     \frac{\Delta x}{2} & 0 & \frac{\partial z}{\partial \xi}\\
     0 & \frac{\Delta y}{2} & \frac{\partial z}{\partial \eta}\\
     0 & 0 & \frac{\partial z}{\partial \zeta}
   \end{array}\right) . \]


Note that $x_i$ and $y_i$ do not appear in partial derivatives of $z$ in the
third column of $J_k$ (see (\ref{eq:mapderivatives})), so equal horizontal
spacing does not let us simplify these. To compute partial derivatives of $z$,
I differentiate (\ref{eq:map}) and get:
\begin{eqnarray}
  \frac{\partial z}{\partial \xi} (\xi, \eta, \zeta) & = & \sum_{i = 1}^8 z_i
  \frac{\partial \chi_i}{\partial \xi} (\xi, \eta, \zeta), \nonumber\\
  \frac{\partial z}{\partial \eta} (\xi, \eta, \zeta) & = & \sum_{i = 1}^8
  z_i  \frac{\partial \chi_i}{\partial \eta} (\xi, \eta, \zeta),
  \label{eq:mapderivatives}\\
  \frac{\partial z}{\partial \zeta} (\xi, \eta, \zeta) & = & \sum_{i = 1}^8
  z_i  \frac{\partial \chi_i}{\partial \zeta} (\xi, \eta, \zeta) . \nonumber
\end{eqnarray}


Note that it is also possible to use (\ref{eq:diffxi}), (\ref{eq:diffeta}), and (\ref{eq:diffzeta}) here (as with derivatives of shape functions, I need to compute the Jacobian \emph{at quadrature points only}).

A matrix of this form is easy to invert by hand:
\begin{equation}
  \left(\begin{array}{lll}
    a_{1, 1} & 0 & a_{1, 3}\\
    0 & a_{2, 2} & a_{2, 3}\\
    0 & 0 & a_{3, 3}
  \end{array} \right)^{- 1} = \left(\begin{array}{ccc}
    \frac{1}{a_{1, 1}} & 0 & - \frac{a_{1, 3}}{a_{1, 1} a_{3, 3}}\\
    0 & \frac{1}{a_{2, 2}} & - \frac{a_{2, 3}}{a_{2, 2} a_{3, 3}}\\
    0 & 0 & \frac{1}{a_{3, 3}}
  \end{array} \right) . \label{eq:jacobianinverse}
\end{equation}


So, pre-computing $| J_k |$ and $J^{-1}_k$ at a quadrature point requires $3$
steps:
\begin{enumerate}
  \item Use (\ref{eq:mapderivatives}) to compute partial derivatives of $z$.

  \item Set $| J_k | = \frac{1}{4} \Delta x \Delta y \frac{\partial
  z}{\partial \zeta} (\xi, \eta, \zeta)$.

  \item Use (\ref{eq:jacobianinverse}) to compute $J_k^{- 1}$.
\end{enumerate}


\begin{thebibliography}{1}
  \bibitem[1]{Elmanetal2005}\textsc{H.~C.~Elman}, \textsc{D.~J.~Silvester}
  and \textsc{A.~J.~Wathen}, \textit{Finite Elements and Fast Iterative
  Solvers: with Applications in Incompressible Fluid Dynamics}, Oxford Univ.
  Press, 2005.{\newblock}

  \bibitem[2]{Kardestuncer1987}\textsc{H.~Kardestuncer},
  \textsc{D.~Norrie} and \textsc{F.~Brezzi}, \textit{Finite element
  handbook}, McGraw-Hill reference books of interest: Handbooks, McGraw-Hill,
  1987.{\newblock}
\end{thebibliography}

\end{document}

% LocalWords: LocalWords Jacobian Khroulev Neumann Galerkin hexahedral hexahedron isoparametric incompressible Kardestuncer Norrie Brezzi invertible dx dy dz quadratures analytical integrands trilinear Maxima precompute precomputing